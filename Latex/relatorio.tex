% primeiro_latex.tex - Meu primeiro documento LATEX

\documentclass{article}

\begin{document}

UnB-Gama \\
Aluno: Luciano Prestes Cavalcanti \\
Disciplina: Introducao a Computacao Grafica \\
\\
Lista 03
\\
\\
a) Descricao da Solucao Proposta
\\
Criacao de classes capazes de gerenciar os detalhes necessários a implementacao
sendo elas color.h, que eh responsavel por gerenciar as variaveis de cor,
point.h para gerenciar os pontos, vector3d.h, para tratar os vetores,
mesh.h, para tratar as malhas triangulares, drawable.h, que eh uma
interface com o metodo draw() que obriga a qualquer coisa que seja
renderizavel ser capaz de se imprimir na tela, ou seja, os objetos
tem a responsabilidade de se auto renderizar dentro da tela,
screen.h, para tratar a renderizacao da tela, e uma main.cpp,
aonde esta implementado o loop principal.
\\
Instrucoes do Programa
\\
1 - Use as SETAS para MOVER o COELHO;
2 - Use PAGEUP para AMPLIAR e PAGEDOWN para REDIZIR o TAMANHO do COELHO;
3 - Pressione L para MOSTRAR o COELHO como LINHAS;
3 - Pressione T para MOSTRAR o COELHO como PLANO;
\\
\\
b) Estrutura de Dados Proposta
\\
classe Drawable, eh uma interface, cujo o unico metodo eh o draw(),
a arquitetura dessa implementacao obriga que qualquer coisa que seja
renderizada na tela deve implementar Drawable, para garantir que todo
objeto a ser renderizado saiba se auto-renderizar.
\\
\\
classe Color, armazena a informacao quanto as cores RGBA.
\\
\\
classe Point3D, armazena as coordenadas de um ponto de tres dimensoes,
ou seja, x, y e z, possui uma instancia de Color, para determinar a cor
no ponto, nao implementa Drawable, porem ele mesmo eh responsavel por
executar as funcoes glColor3d e glVertex3f, tambem eh responsavel por
fazer as operacoes de igualdade, atribuicao e subtracao em relacao a
outros pontos.
\\
\\
classe Vector3D, herda da classe Point3D, porem acrescenta metodos
de multiplicacao por um escalar, magnitude, produto escalar, e
produto vetorial
\\
\\
classe Mesh, responsavel por carregar um arquivo .off e torna-lo
visivel na tela, responsavel tambem por ampliar ou reduzir a imagem,
e reposiciona-la na tela.
\\
\\
classe Screen, responsavel por iniciar o SDL e o OpenGL, determinar
o tamanho da tela, e tambem receber objetos do tipo Drawable para
renderiza-los na tela.
\\
\\
c) Principais Licoes Aprendidas
\\
O formato .off eh um padrao aonde se torna pratico carregar as
informacoes tridimensionais de desenhos. Importante tambem foi o
aprendizado quanto a utilizacao de vetores para redimensionar e
mover objetos em 3D.




\end{document}